\documentclass[14pt, oneside]{altsu-report}

\worktype{Отчёт по практике на тему:}
\title{Создание игры "Морской бой" на языке программирования С++  с использованием библиотеки GTK3.0}
\author{О.\,Р.~Усмонов}
\groupnumber{5.205-1}
\GradebookNumber{1337}
\supervisor{И.\,А.~Шмаков}
\supervisordegree{ст. преп. каф. ВТиЭ}
\ministry{Министерство науки и высшего образования}
\country{Российской Федерации}
\fulluniversityname{ФГБОУ ВО Алтайский государственный университет}
\institute{Институт цифровых технологий, электроники и физики}
\department{Кафедра вычислительной техники и электроники}
\departmentchief{В.\,В.~Пашнев}
\departmentchiefdegree{к.ф.-м.н., доцент}
\shortdepartment{ВТиЭ}
\abstractRU{Большой текст на русском! Пока счётчик работает не правильно! Поправьте количество рисунков и таблиц в cls-файле.}
\abstractEN{Большой текст на английском!}
\keysRU{компьютерное моделирование, cистема управления версиями}
\keysEN{computer simulation, distributed version control}

\date{\the\year}

% Подключение файлов с библиотекой.
\addbibresource{graduate-students.bib}

% Пакет для отладки отступов.
%\usepackage{showframe}

\begin{document}
\maketitle

\setcounter{page}{2}
\makeabstract
\tableofcontents

\chapter*{Введение}
\phantomsection\addcontentsline{toc}{chapter}{ВВЕДЕНИЕ}
\textbf{Актуальность:} hhgggggg 
\textbf{Цель} 
\textbf{Задачи:}
    \begin{enumerate}
        \item Текст много текста.
        \item Текст много текста.
        \item Текст много текста.
    \end{enumerate}

% Подключение первой главы (теория):
\chapter{\label{ch:ch01}ГЛАВА 1} % Нужно сделать главу в содержании заглавными буквами

\section{\label{sec:ch01/sec01}Язык программирования C++}

\subsection{\label{subsec:ch01/sec01/sub01}Введение в язык C++}
Язык программирования C++ является одним из наиболее популярных и востребованных инструментов разработки программного обеспечения. Созданный в 1980-х годах Бьярном Страуструпом как расширение языка C, C++ сочетает в себе преимущества процедурного и объектно-ориентированного программирования. Он отличается высокой производительностью, богатой функциональностью и широким спектром применения.

\subsection{\label{subsec:ch01/sec01/sub02}Синтаксис и основные концепции}
Синтаксис языка C++ базируется на языке C, что делает его привычным для программистов, знакомых с C. Основные концепции C++ включают в себя классы и объекты, которые являются основой объектно-ориентированного программирования. Классы позволяют объединять данные и функции в единый тип, что способствует модульности и повторному использованию кода. Указатели и ссылки являются важными инструментами для работы с памятью и динамическим выделением ресурсов.

\subsection{\label{subsec:ch01/sec01/sub03}Основные возможности языка}
Одной из ключевых особенностей C++ является поддержка наследования и полиморфизма, что позволяет создавать иерархии классов и повторно использовать код. Шаблоны (templates) предоставляют средство для обобщенного программирования, позволяя создавать универсальные функции и классы, работающие с различными типами данных. Исключения позволяют обрабатывать ошибки и исключительные ситуации, обеспечивая более надежное и безопасное выполнение программ.

Стандартная библиотека шаблонов (STL) предоставляет широкий набор контейнеров, алгоритмов и функций для работы с данными, что делает разработку программ быстрой и эффективной.

\subsection{\label{subsec:ch01/sec01/sub04}Многопоточное программирование}
Современные приложения все чаще требуют поддержки многопоточности для эффективной работы с множеством задач параллельно. В C++ многопоточное программирование осуществляется с использованием стандартной библиотеки многопоточности, предоставляющей средства для создания и управления потоками выполнения, синхронизации доступа к общим данным и предотвращения гонок данных.
\subsection{\label{subsec:ch01/sec01/sub05}Примеры приложений и проектов на C++}
C++ широко используется в различных областях, включая разработку игр, программирование микроконтроллеров, системное программирование, создание операционных систем и многое другое. Известные проекты, разработанные на C++, включают операционные системы Linux и Windows, браузер Google Chrome, игры из серии Unreal Engine и многие другие.
\\
\\
Язык программирования C++ остается актуальным и востребованным инструментом разработки благодаря своей мощности, гибкости и высокой производительности. С постоянным развитием стандарта и расширением функциональности, C++ продолжает оставаться одним из лучших выборов для создания сложных и производительных программных продуктов.


\section{\label{sec:ch01/sec02}Библиотека GTK+}

\subsection{\label{subsec:ch01/sec02/sub01}Введение в библиотеку GTK+}
Библиотека GTK+ (GIMP Toolkit) является одной из самых популярных и мощных библиотек для создания графических пользовательских интерфейсов (GUI) в приложениях под операционные системы Linux, Unix и другие UNIX-подобные системы. Она была создана для использования в графическом редакторе GIMP, но в настоящее время широко применяется в различных проектах, включая графические редакторы, офисные приложения, мультимедийные проигрыватели и многие другие.

\subsection{\label{subsec:ch01/sec02/sub02}Синтаксис и основные концепции}

GTK+ предоставляет разработчикам широкий набор виджетов (кнопки, поля ввода, таблицы и т. д.) и инструментов для создания интерактивных пользовательских интерфейсов. Основными концепциями GTK+ являются сигналы и обработчики событий, которые позволяют приложениям реагировать на действия пользователя. GTK+ также поддерживает каскадные таблицы стилей (CSS), что обеспечивает гибкую настройку внешнего вида интерфейса.


\subsection{\label{subsec:ch01/sec02/sub03}Основные возможности библиотеки}
Одной из ключевых особенностей библиотеки GTK+ является её расширяемость и гибкость. GTK+ предоставляет разработчикам не только набор стандартных виджетов, но и возможность создания собственных пользовательских элементов интерфейса. Это достигается с помощью механизма наследования и композиции, позволяя создавать новые виджеты, которые полностью интегрируются в окружение GTK+ и могут быть использованы вместе с уже существующими элементами.

Благодаря поддержке каскадных таблиц стилей (CSS), GTK+ обеспечивает простую и гибкую настройку внешнего вида приложений. Разработчики могут легко изменять цвета, шрифты, отступы и другие атрибуты виджетов, а также создавать собственные стили, чтобы придать своим приложениям уникальный и профессиональный вид.

Еще одной важной особенностью GTK+ является её междисциплинарность. Библиотека активно используется не только в области разработки программного обеспечения, но и в области дизайна и создания графических интерфейсов. Для дизайнеров GTK+ предоставляет инструменты для создания прототипов интерфейсов, которые могут быть легко протестированы и переданы разработчикам для реализации.

Более того, GTK+ предлагает широкий набор инструментов для мультимедийного программирования. Это включает в себя возможности работы с аудио и видео файлами, мультимедийными потоками и графическими элементами. Разработчики могут создавать мощные и интерактивные мультимедийные приложения, используя функциональность GTK+ в сочетании с другими библиотеками и инструментами.

В целом, GTK+ представляет собой полноценный инструментарий для создания современных и профессиональных графических пользовательских интерфейсов в приложениях под операционные системы Linux, Unix и другие UNIX-подобные системы. Её гибкость, расширяемость и многофункциональность делают её популярным выбором среди разработчиков и дизайнеров софта.
\\
\\
Библиотека GTK+ является мощным инструментом для создания кроссплатформенных графических пользовательских интерфейсов в приложениях под операционные системы Linux и Unix. Ее гибкость, производительность и богатый набор функций делают ее популярным выбором среди разработчиков софта, стремящихся создавать современные и удобные для пользователей приложения.

\section{\label{sec:ch01/sec03}Система контроля версий Git}

\subsection{\label{subsec:ch01/sec01/sub01}Введение в Git}
Система контроля версий (СКВ) играет ключевую роль в разработке программного обеспечения, обеспечивая эффективное управление изменениями в коде, отслеживание истории версий и совместную работу разработчиков. Среди различных инструментов СКВ выделяется Git, который отличается высокой производительностью, гибкостью и распределенной архитектурой. Созданный Линусом Торвальдсом в 2005 году, Git быстро стал стандартом в индустрии разработки программного обеспечения благодаря своей простоте использования и мощным возможностям.

\subsection{\label{subsec:ch01/sec01/sub02}Принципы работы Git} 
Принципы работы Git основаны на его распределенной архитектуре и концепции контроля версий. Вот более подробное описание ключевых принципов работы Git:
\begin{enumerate}
    \item \textbf{Репозиторий:}  В центре работы Git находится репозиторий, который представляет собой базу данных, содержащую все версии файлов и историю изменений проекта. Репозиторий может быть размещен на локальном компьютере или удаленном сервере, и каждый участник проекта имеет к нему доступ.
    \item \textbf{Распределенная архитектура:} Одной из ключевых особенностей Git является его распределенная архитектура. Это означает, что каждый участник проекта имеет полную копию репозитория со всей историей изменений. Это обеспечивает высокую отказоустойчивость и позволяет разработчикам работать независимо друг от друга, сохраняя при этом возможность синхронизации своих изменений с общим репозиторием.
    \item \textbf{Коммиты:} Коммиты представляют собой сохраненные изменения в репозитории. Каждый коммит содержит информацию о внесенных изменениях, авторе и времени создания. Коммиты служат для отслеживания истории изменений проекта и обеспечивают возможность отката к предыдущим версиям кода.
    \item \textbf{Ветвление и слияние:} Git позволяет создавать отдельные ветки для работы над определенными функциями или исправлениями. Каждая ветка представляет собой отдельную линию разработки, в которой можно вносить изменения независимо от основной ветки. После завершения работы в ветке изменения могут быть слиты с основной веткой с помощью процесса слияния.
    \item \textbf{История изменений и откат к предыдущим версиям:} Git сохраняет историю изменений проекта, позволяя разработчикам отслеживать, кто, когда и какие изменения вносил в код. Это обеспечивает прозрачность и отслеживаемость процесса разработки. Кроме того, благодаря системе коммитов и ветвлений, можно легко откатиться к предыдущим версиям кода в случае необходимости.
    \\
    \\
    Эти принципы работы Git обеспечивают гибкость, скорость и надежность в управлении версиями кода, что делает его одним из самых популярных инструментов в индустрии разработки программного обеспечения.
\end{enumerate}


\subsection{\label{subsec:ch01/sec01/sub03}Основные возможности Git}
Основные возможности Git представляют собой набор функций и инструментов, которые делают его одним из самых мощных и гибких инструментов управления версиями кода. Вот более подробное описание ключевых возможностей Git:
\begin{enumerate}
    \item \textbf{Ветвление (Branching):} Одной из наиболее мощных возможностей Git является ветвление. Разработчики могут создавать отдельные ветки для работы над определенными функциями или исправлениями, не затрагивая основную ветку разработки. Это позволяет проводить эксперименты, тестировать новый функционал и вносить изменения без риска нарушения стабильности основного кода.
    \item \textbf{Слияние (Merging):} После завершения работы в отдельной ветке, изменения могут быть объединены с основной веткой разработки с помощью процесса слияния. Git автоматически обнаруживает и разрешает конфликты, возникающие при слиянии изменений из разных веток, обеспечивая гладкое и безопасное объединение кода.
    \item \textbf{Коммиты (Commits):} Каждое изменение в коде сохраняется в Git как коммит. Коммиты представляют собой логические единицы работы, которые содержат информацию о внесенных изменениях, авторе и времени создания. Это позволяет разработчикам отслеживать историю изменений проекта и возвращаться к предыдущим версиям кода при необходимости.
    \item \textbf{История изменений (History):} Git сохраняет историю изменений проекта, что позволяет разработчикам отслеживать, кто, когда и какие изменения вносил в код. История изменений предоставляет прозрачность и отслеживаемость процесса разработки, а также обеспечивает возможность проверки и аудита внесенных изменений.
    \item \textbf{Откат к предыдущим версиям (Rollback):} В случае необходимости разработчики могут легко откатиться к предыдущим версиям кода с помощью Git. Это особенно полезно в случаях, когда внесенные изменения привели к ошибкам или неожиданным последствиям, и требуется быстрое восстановление стабильного состояния проекта.
    \item \textbf{Работа с удаленными репозиториями (Remote Repositories):} Git поддерживает работу с удаленными репозиториями, что позволяет разработчикам сотрудничать над проектами удаленно. Они могут загружать свои изменения на удаленный сервер, скачивать изменения других участников проекта и обмениваться обновлениями кода, обеспечивая эффективную командную работу.
    \item \textbf{Конфигурация и настройка (Configuration):} Git предоставляет разнообразные возможности конфигурации и настройки, которые позволяют настраивать его поведение в соответствии с потребностями проекта и команды разработчиков. Это включает в себя настройку правил игнорирования файлов, настройку внешних инструментов сравнения и слияния, а также настройку пользовательских алиасов для выполнения часто используемых команд.\\
    Эти основные возможности Git делают его незаменимым инструментом в управлении версиями кода, обеспечивая высокую скорость, гибкость и надежность в процессе разработки программного обеспечения.
\end{enumerate}

\subsection{\label{subsec:ch01/sec01/sub04}Применение Git в разработке ПО}
Git широко применяется в разработке программного обеспечения всех масштабов - от небольших персональных проектов до крупных корпоративных приложений. Множество крупных компаний, таких как Google, Facebook, Microsoft и многие другие, используют Git в своих проектах и вкладывают ресурсы в его развитие и поддержку.

Git обеспечивает высокую скорость и эффективность разработки, позволяя разработчикам работать над проектами параллельно, отслеживать изменения и легко совмещать свою работу с работой других участников команды. Благодаря его распределенной архитектуре и гибкости, Git подходит для работы в различных сценариях разработки, включая разработку открытого и закрытого исходного кода, а также для сотрудничества между разработчиками по всему миру.
% Подключение второй главы (практическая часть):
\chapter{\label{ch:ch02}ГЛАВА 2}


Пример использования minted для оформления кода и ссылка на этот код~\ref{code:example04}.
\begin{code}
\captionof{listing}{\centering\label{code:example04}Сложение двух массивов параллельно десятью потоками (пример из https://ru.wikipedia.org/wiki/OpenMP)}
\vspace{-\baselineskip}\begin{minted}{C}
#include <stdio.h>
#include <omp.h>
#define N 100

int main(int argc, char *argv[]) {
  double a[N], b[N], c[N];
  int i;
  omp_set_dynamic(0); // запретить библиотеке openmp менять число потоков во время исполнения
  omp_set_num_threads(10); // установить число потоков в 10
  // инициализируем массивы
  for (i = 0; i < N; i++) {
      a[i] = i * 1.0;
      b[i] = i * 2.0;
  }
  // вычисляем сумму массивов
#pragma omp parallel for shared(a, b, c) private(i)
   for (i = 0; i < N; i++)
     c[i] = a[i] + b[i];

  printf ("%f\n", c[10]);
  return 0;
}
\end{minted}
\end{code}

% Подключение третий главы (практическая часть с тестированием:


\chapter*{Заключение}
\phantomsection\addcontentsline{toc}{chapter}{ЗАКЛЮЧЕНИЕ}

    \begin{enumerate}
        \item Пример ссылки на электронный источник~\cite{wikiRUBitbucket,wikiRUIdSoftware,wikiRUGitHub,HABR,Stack}.
        \item Пример ссылки на книгу одного автора~\cite{book1author}.
        \item Пример ссылки на книгу 5-ти и более авторов~\cite{book5author}.
    \end{enumerate}

\newpage
\phantomsection\addcontentsline{toc}{chapter}{СПИСОК ИСПОЛЬЗОВАННОЙ ЛИТЕРАТУРЫ}
\printbibliography[title={Список использованной литературы}]

\appendix
\newpage
\chapter*{\raggedleft\label{appendix1}Приложение}
\phantomsection\addcontentsline{toc}{chapter}{ПРИЛОЖЕНИЕ}
%\section*{\centering\label{code:appendix}Текст программы}

\begin{center}
    \label{code:appendix}Текст программы
\end{center}

\begin{code}
\captionof*{listing}{\centering\label{code:pi-example}Пример программы вычисления числа $\pi$ на языке \textit{C} с использованием \textit{MPI} (пример из https://ru.wikipedia
.org/wiki/Message\_Passing\_Interface)}
\vspace{-1cm}\inputminted{C}{src/pi-mpi.c}
\end{code}

\end{document}

